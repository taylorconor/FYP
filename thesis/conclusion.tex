\section{Future work}
There are many worthwhile areas for future work, because to the best of the author's knowledge, this is the first attempted design of a fully decentralised location proof system.

\subsection{Verification techniques}
Section \ref{ssec:verification} outlined a number of simple verification techniques that a Verifier node could use to analyse the a proof chain. There is potential for further research on advanced techniques of analysing proof chain trees to determine the validity of a user's proof request.

\subsection{Privacy and verifiability}
As explained in section \ref{sssec:key_packets}, certain transactions can be hidden from a proof chain in order to protect a user's privacy, but maintain chronological integrity. There is an interesting area of further work in studying the effect that hiding certain proofs has on that user's verifiability.

\subsection{Sybil attack}
Previous work has indicated that there is no provably correct way of preventing a Sybil attack in a decentralised system \cite{sybil}. However, further work studying strong mitigations against a Sybil attack, or even preventing a Sybil attack entirely, would greatly improve the appeal of a decentralised location proof system, and potentially lead to its ubiquitous use.

\section{Conclusion}
This project presented the design of a decentralised location proof system. The aim of the project was to design a system that satisfied the design goals, as defined in section \ref{sec:design_goals}. This system satisfies all of the requirements of OTIT \cite{otit}, and all but one of the threats defined in section \ref{ssec:threats}. However, chapter \ref{ch:evaluation} concluded that this Sybil attack threat must be highly targeted to be effective.

In conclusion, the model presented in this project is a viable design of a decentralised location proof system, providing that the Sybil attack vulnerability can be heavily mitigated against or prevented outright.